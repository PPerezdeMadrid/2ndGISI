% Options for packages loaded elsewhere
\PassOptionsToPackage{unicode}{hyperref}
\PassOptionsToPackage{hyphens}{url}
%
\documentclass[
]{article}
\usepackage{amsmath,amssymb}
\usepackage{lmodern}
\usepackage{iftex}
\ifPDFTeX
  \usepackage[T1]{fontenc}
  \usepackage[utf8]{inputenc}
  \usepackage{textcomp} % provide euro and other symbols
\else % if luatex or xetex
  \usepackage{unicode-math}
  \defaultfontfeatures{Scale=MatchLowercase}
  \defaultfontfeatures[\rmfamily]{Ligatures=TeX,Scale=1}
\fi
% Use upquote if available, for straight quotes in verbatim environments
\IfFileExists{upquote.sty}{\usepackage{upquote}}{}
\IfFileExists{microtype.sty}{% use microtype if available
  \usepackage[]{microtype}
  \UseMicrotypeSet[protrusion]{basicmath} % disable protrusion for tt fonts
}{}
\makeatletter
\@ifundefined{KOMAClassName}{% if non-KOMA class
  \IfFileExists{parskip.sty}{%
    \usepackage{parskip}
  }{% else
    \setlength{\parindent}{0pt}
    \setlength{\parskip}{6pt plus 2pt minus 1pt}}
}{% if KOMA class
  \KOMAoptions{parskip=half}}
\makeatother
\usepackage{xcolor}
\usepackage[margin=1in]{geometry}
\usepackage{color}
\usepackage{fancyvrb}
\newcommand{\VerbBar}{|}
\newcommand{\VERB}{\Verb[commandchars=\\\{\}]}
\DefineVerbatimEnvironment{Highlighting}{Verbatim}{commandchars=\\\{\}}
% Add ',fontsize=\small' for more characters per line
\usepackage{framed}
\definecolor{shadecolor}{RGB}{248,248,248}
\newenvironment{Shaded}{\begin{snugshade}}{\end{snugshade}}
\newcommand{\AlertTok}[1]{\textcolor[rgb]{0.94,0.16,0.16}{#1}}
\newcommand{\AnnotationTok}[1]{\textcolor[rgb]{0.56,0.35,0.01}{\textbf{\textit{#1}}}}
\newcommand{\AttributeTok}[1]{\textcolor[rgb]{0.77,0.63,0.00}{#1}}
\newcommand{\BaseNTok}[1]{\textcolor[rgb]{0.00,0.00,0.81}{#1}}
\newcommand{\BuiltInTok}[1]{#1}
\newcommand{\CharTok}[1]{\textcolor[rgb]{0.31,0.60,0.02}{#1}}
\newcommand{\CommentTok}[1]{\textcolor[rgb]{0.56,0.35,0.01}{\textit{#1}}}
\newcommand{\CommentVarTok}[1]{\textcolor[rgb]{0.56,0.35,0.01}{\textbf{\textit{#1}}}}
\newcommand{\ConstantTok}[1]{\textcolor[rgb]{0.00,0.00,0.00}{#1}}
\newcommand{\ControlFlowTok}[1]{\textcolor[rgb]{0.13,0.29,0.53}{\textbf{#1}}}
\newcommand{\DataTypeTok}[1]{\textcolor[rgb]{0.13,0.29,0.53}{#1}}
\newcommand{\DecValTok}[1]{\textcolor[rgb]{0.00,0.00,0.81}{#1}}
\newcommand{\DocumentationTok}[1]{\textcolor[rgb]{0.56,0.35,0.01}{\textbf{\textit{#1}}}}
\newcommand{\ErrorTok}[1]{\textcolor[rgb]{0.64,0.00,0.00}{\textbf{#1}}}
\newcommand{\ExtensionTok}[1]{#1}
\newcommand{\FloatTok}[1]{\textcolor[rgb]{0.00,0.00,0.81}{#1}}
\newcommand{\FunctionTok}[1]{\textcolor[rgb]{0.00,0.00,0.00}{#1}}
\newcommand{\ImportTok}[1]{#1}
\newcommand{\InformationTok}[1]{\textcolor[rgb]{0.56,0.35,0.01}{\textbf{\textit{#1}}}}
\newcommand{\KeywordTok}[1]{\textcolor[rgb]{0.13,0.29,0.53}{\textbf{#1}}}
\newcommand{\NormalTok}[1]{#1}
\newcommand{\OperatorTok}[1]{\textcolor[rgb]{0.81,0.36,0.00}{\textbf{#1}}}
\newcommand{\OtherTok}[1]{\textcolor[rgb]{0.56,0.35,0.01}{#1}}
\newcommand{\PreprocessorTok}[1]{\textcolor[rgb]{0.56,0.35,0.01}{\textit{#1}}}
\newcommand{\RegionMarkerTok}[1]{#1}
\newcommand{\SpecialCharTok}[1]{\textcolor[rgb]{0.00,0.00,0.00}{#1}}
\newcommand{\SpecialStringTok}[1]{\textcolor[rgb]{0.31,0.60,0.02}{#1}}
\newcommand{\StringTok}[1]{\textcolor[rgb]{0.31,0.60,0.02}{#1}}
\newcommand{\VariableTok}[1]{\textcolor[rgb]{0.00,0.00,0.00}{#1}}
\newcommand{\VerbatimStringTok}[1]{\textcolor[rgb]{0.31,0.60,0.02}{#1}}
\newcommand{\WarningTok}[1]{\textcolor[rgb]{0.56,0.35,0.01}{\textbf{\textit{#1}}}}
\usepackage{longtable,booktabs,array}
\usepackage{calc} % for calculating minipage widths
% Correct order of tables after \paragraph or \subparagraph
\usepackage{etoolbox}
\makeatletter
\patchcmd\longtable{\par}{\if@noskipsec\mbox{}\fi\par}{}{}
\makeatother
% Allow footnotes in longtable head/foot
\IfFileExists{footnotehyper.sty}{\usepackage{footnotehyper}}{\usepackage{footnote}}
\makesavenoteenv{longtable}
\usepackage{graphicx}
\makeatletter
\def\maxwidth{\ifdim\Gin@nat@width>\linewidth\linewidth\else\Gin@nat@width\fi}
\def\maxheight{\ifdim\Gin@nat@height>\textheight\textheight\else\Gin@nat@height\fi}
\makeatother
% Scale images if necessary, so that they will not overflow the page
% margins by default, and it is still possible to overwrite the defaults
% using explicit options in \includegraphics[width, height, ...]{}
\setkeys{Gin}{width=\maxwidth,height=\maxheight,keepaspectratio}
% Set default figure placement to htbp
\makeatletter
\def\fps@figure{htbp}
\makeatother
\setlength{\emergencystretch}{3em} % prevent overfull lines
\providecommand{\tightlist}{%
  \setlength{\itemsep}{0pt}\setlength{\parskip}{0pt}}
\setcounter{secnumdepth}{-\maxdimen} % remove section numbering
\ifLuaTeX
  \usepackage{selnolig}  % disable illegal ligatures
\fi
\IfFileExists{bookmark.sty}{\usepackage{bookmark}}{\usepackage{hyperref}}
\IfFileExists{xurl.sty}{\usepackage{xurl}}{} % add URL line breaks if available
\urlstyle{same} % disable monospaced font for URLs
\hypersetup{
  pdftitle={3-Variables aleatorias I},
  hidelinks,
  pdfcreator={LaTeX via pandoc}}

\title{3-Variables aleatorias I}
\author{}
\date{\vspace{-2.5em}}

\begin{document}
\maketitle

\hypertarget{distribuciones-de-variables-aleatorias-discretas}{%
\section{Distribuciones de variables aleatorias
discretas}\label{distribuciones-de-variables-aleatorias-discretas}}

\begin{center}\rule{0.5\linewidth}{0.5pt}\end{center}

Sea la VA X: ``nº de caras en n lanzamientos de una moneda cuya
probabilidad de cara es p''. Estudia su distribución para el caso
\(p=1/2\), \(n=100\) mediante la función de probabilidad.

\[P(X=x) = nCx\cdot p^x \cdot (1-p)^(n-x)\]

\begin{Shaded}
\begin{Highlighting}[]
\NormalTok{p\_heads }\OtherTok{=} \ControlFlowTok{function}\NormalTok{(x , n, p)\{}
  \FunctionTok{choose}\NormalTok{(n , x) }\SpecialCharTok{*}\NormalTok{ p }\SpecialCharTok{\^{}}\NormalTok{ x }\SpecialCharTok{*}\NormalTok{ (}\DecValTok{1}\SpecialCharTok{{-}}\NormalTok{p)}\SpecialCharTok{\^{}}\NormalTok{(n}\SpecialCharTok{{-}}\NormalTok{x)  }
\NormalTok{\}}

\CommentTok{\#calcular todas las probabilidades desde x = 0 hasta x = 100}
\NormalTok{  all\_probs }\OtherTok{=} \FunctionTok{p\_heads}\NormalTok{(}\DecValTok{0}\SpecialCharTok{:}\DecValTok{100}\NormalTok{ , }\DecValTok{100}\NormalTok{, }\FloatTok{0.5}\NormalTok{)}
  \FunctionTok{plot}\NormalTok{(}\DecValTok{0}\SpecialCharTok{:}\DecValTok{100}\NormalTok{, all\_probs) }\CommentTok{\#dibujar gráfica}
\end{Highlighting}
\end{Shaded}

\includegraphics{3-random_vars_p1_statements_files/figure-latex/unnamed-chunk-1-1.pdf}

\begin{Shaded}
\begin{Highlighting}[]
  \FunctionTok{plot}\NormalTok{(}\DecValTok{0}\SpecialCharTok{:}\DecValTok{100}\NormalTok{, all\_probs, }\AttributeTok{type =} \StringTok{"h"}\NormalTok{, }\AttributeTok{xlab =} \StringTok{"x: nº de caras"}\NormalTok{ , }\AttributeTok{ylab =} \StringTok{"P(X=x)"}\NormalTok{)}
\end{Highlighting}
\end{Shaded}

\includegraphics{3-random_vars_p1_statements_files/figure-latex/unnamed-chunk-1-2.pdf}

\begin{center}\rule{0.5\linewidth}{0.5pt}\end{center}

Escribe una función de R para la función de distribución de la VA
aleatoria X: '\,'nº de caras en 100 lanzamientos de una moneda cuya
probabilidad de cara es 0.5''. Dibújala y úsala para responder a las
siguientes preguntas:

\begin{itemize}
\tightlist
\item
  \[ P(40 \leq X \leq 60) \]
\item
  \[ P(40 < X < 60) \]
\item
  \[ P(40 \leq X < 60)\]
\end{itemize}

\begin{Shaded}
\begin{Highlighting}[]
\CommentTok{\#P(40\textless{}= x \textless{}= 60) ==\textgreater{}\textgreater{}  Print(F(60){-}F(39))}

\NormalTok{F\_heads }\OtherTok{=} \ControlFlowTok{function}\NormalTok{(x , n , p)\{}
  \CommentTok{\#F(x) = P(X \textless{}= x )= sum(all P(X = x) X \textless{}= x)}
  \FunctionTok{sum}\NormalTok{(}\FunctionTok{p\_heads}\NormalTok{(}\DecValTok{0}\SpecialCharTok{:}\NormalTok{x,n,p))     }\CommentTok{\#calcular probabildades del cero al x \textgreater{}\textgreater{}\textgreater{}\textgreater{} 0:x}
\NormalTok{\}}

\CommentTok{\#FUNCION NO ESTÁ BIEN VECTORIZADA }
  \CommentTok{\# x = 1:2 {-}{-}{-}{-}\textgreater{} error!!}
  \CommentTok{\# solución {-}{-}\textgreater{} Vectorize::= recibe funciones y devuleve una nueva funcion vectorizada}
\NormalTok{  F\_heads }\OtherTok{=} \FunctionTok{Vectorize}\NormalTok{(F\_heads)}

\CommentTok{\#P(40\textless{}= x \textless{}= 60)}
\FunctionTok{print}\NormalTok{(}\FunctionTok{F\_heads}\NormalTok{(}\DecValTok{60}\NormalTok{, }\DecValTok{100}\NormalTok{, }\FloatTok{0.5}\NormalTok{)}\SpecialCharTok{{-}}\FunctionTok{F\_heads}\NormalTok{(}\DecValTok{39}\NormalTok{, }\DecValTok{100}\NormalTok{, }\FloatTok{0.5}\NormalTok{))}
\end{Highlighting}
\end{Shaded}

\begin{verbatim}
## [1] 0.9647998
\end{verbatim}

\begin{Shaded}
\begin{Highlighting}[]
\CommentTok{\#P(40 \textless{} x \textless{} 60) = F(59) {-} F(40)}
\FunctionTok{print}\NormalTok{(}\FunctionTok{F\_heads}\NormalTok{(}\DecValTok{59}\NormalTok{, }\DecValTok{100}\NormalTok{, }\FloatTok{0.5}\NormalTok{)}\SpecialCharTok{{-}} \FunctionTok{F\_heads}\NormalTok{(}\DecValTok{40}\NormalTok{, }\DecValTok{100}\NormalTok{, }\FloatTok{0.5}\NormalTok{))}
\end{Highlighting}
\end{Shaded}

\begin{verbatim}
## [1] 0.9431121
\end{verbatim}

\begin{Shaded}
\begin{Highlighting}[]
\CommentTok{\#P(40\textless{}= x \textless{} 60) = F(59){-}F(39)}
\FunctionTok{print}\NormalTok{(}\FunctionTok{F\_heads}\NormalTok{(}\DecValTok{59}\NormalTok{, }\DecValTok{100}\NormalTok{, }\FloatTok{0.5}\NormalTok{)}\SpecialCharTok{{-}} \FunctionTok{F\_heads}\NormalTok{(}\DecValTok{39}\NormalTok{, }\DecValTok{100}\NormalTok{, }\FloatTok{0.5}\NormalTok{))}
\end{Highlighting}
\end{Shaded}

\begin{verbatim}
## [1] 0.9539559
\end{verbatim}

\begin{Shaded}
\begin{Highlighting}[]
\CommentTok{\#P(X\textgreater{}20) = 1 {-} P(X \textless{}= 20) = 1 {-} F(20)}

\CommentTok{\#GRÁFICO DE LA FUNCIÓN (solo se puede hacer si se vectoriza la f(x))}
\NormalTok{  F\_values }\OtherTok{=} \FunctionTok{F\_heads}\NormalTok{(}\DecValTok{0}\SpecialCharTok{:}\DecValTok{100}\NormalTok{, }\DecValTok{100}\NormalTok{, }\FloatTok{0.5}\NormalTok{)}
  \FunctionTok{plot}\NormalTok{(}\DecValTok{0}\SpecialCharTok{:}\DecValTok{100}\NormalTok{, F\_values, )}
\end{Highlighting}
\end{Shaded}

\includegraphics{3-random_vars_p1_statements_files/figure-latex/unnamed-chunk-2-1.pdf}

\begin{Shaded}
\begin{Highlighting}[]
  \FunctionTok{plot}\NormalTok{(}\DecValTok{0}\SpecialCharTok{:}\DecValTok{100}\NormalTok{, F\_values, }\AttributeTok{type =} \StringTok{"s"}\NormalTok{, }\AttributeTok{xlab=} \StringTok{" x: nº caras [ H ] "}\NormalTok{, }\AttributeTok{ylab =} \StringTok{"F(X)"}\NormalTok{, }
       \AttributeTok{main =} \StringTok{"Función de distribución  o de probabilidad acumulada"}\NormalTok{)}
\end{Highlighting}
\end{Shaded}

\includegraphics{3-random_vars_p1_statements_files/figure-latex/unnamed-chunk-2-2.pdf}

\begin{center}\rule{0.5\linewidth}{0.5pt}\end{center}

Halla la función de probabilidad de X: `'nº de caras en 2 lanzamientos
de una moneda cuya probabilidad de cara es 0.5'' a partir de la
siguiente función de distribución:

\begin{Shaded}
\begin{Highlighting}[]
\NormalTok{F\_values }\OtherTok{=} \FunctionTok{c}\NormalTok{(}\StringTok{\textquotesingle{}0\textquotesingle{}} \OtherTok{=} \FloatTok{0.125}\NormalTok{, }\StringTok{\textquotesingle{}1\textquotesingle{}} \OtherTok{=} \FloatTok{0.5}\NormalTok{, }\StringTok{\textquotesingle{}2\textquotesingle{}} \OtherTok{=} \FloatTok{0.875}\NormalTok{, }\StringTok{\textquotesingle{}3\textquotesingle{}} \OtherTok{=} \DecValTok{1}\NormalTok{)}
\end{Highlighting}
\end{Shaded}

\hypertarget{esperanza-varianza-y-otros-estaduxedsticos-resumen}{%
\section{Esperanza varianza y otros estadísticos
resumen}\label{esperanza-varianza-y-otros-estaduxedsticos-resumen}}

\hypertarget{medidas-de-tendencia-central}{%
\subsection{Medidas de tendencia
central}\label{medidas-de-tendencia-central}}

\begin{center}\rule{0.5\linewidth}{0.5pt}\end{center}

La mediana y la moda de X (`'nº de caras en 100 lanzamientos de una
moneda sin trucar'') son fáciles de calcular por razonamiento. Realiza
esos mismos cálculos usando R.

\begin{Shaded}
\begin{Highlighting}[]
\CommentTok{\# ???}
\end{Highlighting}
\end{Shaded}

\begin{center}\rule{0.5\linewidth}{0.5pt}\end{center}

Calcula la esperanza de la variable aleatoria X:`'nº de caras en 100
lanzamientos de una moneda sin trucar'' usando 1) la definición y 2)
simulaciones. ¿Cuadra con tu intuición?

\begin{Shaded}
\begin{Highlighting}[]
\CommentTok{\# Resultado teórico}
  \CommentTok{\#E[X] = mu = 0 * p(0) + 1*p(1) + 2*p(2) + ... +100*p(100) = sum xi * p(xi)}
  \CommentTok{\#0 * p\_heads (0, 100){-}{-}\textgreater{}  probabilidad de q salga 0 caras con 100 lanzamientos, P(cara) = 0.5}
  \CommentTok{\#0*p\_heads(0, 100, 0.5) + 1*p\_heads(1, 100, 0.5)+...}
   \FunctionTok{sum}\NormalTok{(}\DecValTok{0}\SpecialCharTok{:}\DecValTok{100} \SpecialCharTok{*} \FunctionTok{p\_heads}\NormalTok{(}\DecValTok{0}\SpecialCharTok{:}\DecValTok{100}\NormalTok{ , }\AttributeTok{n =} \DecValTok{100}\NormalTok{, }\AttributeTok{p =} \FloatTok{0.5}\NormalTok{))}
\end{Highlighting}
\end{Shaded}

\begin{verbatim}
## [1] 50
\end{verbatim}

\begin{Shaded}
\begin{Highlighting}[]
\CommentTok{\# 2) Simulaciones }
    \CommentTok{\#Opcion replicate}
    \FunctionTok{replicate}\NormalTok{(}\DecValTok{100}\NormalTok{, }\FunctionTok{sample}\NormalTok{(}\DecValTok{1}\SpecialCharTok{:}\DecValTok{2}\NormalTok{, }\DecValTok{1}\NormalTok{)) }
\end{Highlighting}
\end{Shaded}

\begin{verbatim}
##   [1] 1 2 2 1 1 1 1 1 2 2 1 2 1 2 2 1 2 2 1 1 1 2 1 2 2 1 2 2 1 1 1 1 2 2 1 2 2
##  [38] 2 2 2 2 1 2 2 2 1 2 2 1 1 1 2 2 2 2 1 2 1 1 1 1 2 2 2 2 2 1 2 2 2 2 1 1 2
##  [75] 2 1 2 2 2 2 2 1 2 1 2 2 1 2 2 2 1 1 1 1 2 1 2 1 1 1
\end{verbatim}

\begin{Shaded}
\begin{Highlighting}[]
    \CommentTok{\#Opcion sample}
    \FunctionTok{sample}\NormalTok{(}\DecValTok{1}\SpecialCharTok{:}\DecValTok{2}\NormalTok{, }\DecValTok{100}\NormalTok{, }\AttributeTok{replace =} \StringTok{"TRUE"}\NormalTok{)}
\end{Highlighting}
\end{Shaded}

\begin{verbatim}
##   [1] 1 2 2 1 1 1 1 2 2 2 2 2 1 1 1 1 2 2 2 1 1 2 1 2 2 2 2 2 2 1 2 1 1 1 1 2 1
##  [38] 1 2 2 1 1 2 1 2 2 1 2 1 1 1 1 2 2 2 2 2 2 2 1 2 1 2 2 2 1 2 2 1 1 1 1 2 1
##  [75] 1 1 2 1 1 2 2 2 2 2 1 1 1 1 2 2 2 2 2 1 2 2 1 2 1 2
\end{verbatim}

\begin{Shaded}
\begin{Highlighting}[]
    \CommentTok{\#cuenta cuantos 1´s hay y cuantos 2´s }
    \FunctionTok{sum}\NormalTok{(}\FunctionTok{sample}\NormalTok{(}\DecValTok{0}\SpecialCharTok{:}\DecValTok{1}\NormalTok{, }\DecValTok{100}\NormalTok{, }\AttributeTok{replace =} \StringTok{"TRUE"}\NormalTok{))}
\end{Highlighting}
\end{Shaded}

\begin{verbatim}
## [1] 51
\end{verbatim}

\begin{Shaded}
\begin{Highlighting}[]
\NormalTok{    N }\OtherTok{=} \DecValTok{5000}
\NormalTok{   sims }\OtherTok{=}  \FunctionTok{replicate}\NormalTok{(N, \{}
      \FunctionTok{sum}\NormalTok{(}\FunctionTok{sample}\NormalTok{(}\DecValTok{0}\SpecialCharTok{:}\DecValTok{1}\NormalTok{, }\DecValTok{100}\NormalTok{, }\AttributeTok{replace =} \StringTok{"TRUE"}\NormalTok{))}
\NormalTok{    \})}
   
\CommentTok{\#TEOREMA DE LOS NUMEROS GRANDES}
  \FunctionTok{print}\NormalTok{(}
    \CommentTok{\#sum(sims) / N }
    \FunctionTok{mean}\NormalTok{(sims)  }\CommentTok{\#mean implementa la media muestral}
\NormalTok{  )}
\end{Highlighting}
\end{Shaded}

\begin{verbatim}
## [1] 49.9382
\end{verbatim}

\begin{longtable}[]{@{}
  >{\raggedright\arraybackslash}p{(\columnwidth - 0\tabcolsep) * \real{0.0556}}@{}}
\toprule()
\endhead
Calcula la varianza y la desviación típica de la variable aleatoria X:
``nº de caras en 100 lanzamientos de una moneda sin trucar'' usando 1)
la definición y 2) simulaciones. Visualiza la desviación típica sobre la
función de probabildad \\
\texttt{r\ \#1)\ DEFINICIÓN:\ \#\ Var{[}X{]}\ =\ sum\ (x\_i\ -\ \textbackslash{}mu)\^{}2\ *\ P(X=xi)\ \#\ VAR{[}X{]}\ =\ (0\ -\ mu)\^{}2\ *\ P(X\ =\ 0)\ +\ (1\ -mu)\^{}2\ *\ P(X=1)\ +\ ...\ (mu\ =\ 50,\ calculado\ antes)\ \#\ E{[}X{]}\ =\ sum(0:100\ *\ p\_heads(0:100\ ,\ n\ =\ 100,\ p\ =\ 0.5))\ var\_heads\ =\ sum(\ (0:100\ -\ 50)\^{}2\ *\ p\_heads(0:100,\ 100,\ 0.5)\ )\ print(var\_heads)} \\
\texttt{\#\#\ {[}1{]}\ 25} \\
\texttt{r\ \#desviación\ típica\ (√varianza)\ sd\_heads\ =\ sqrt(var\_heads)\ print(sd\_heads)} \\
\texttt{\#\#\ {[}1{]}\ 5} \\
```r \#VISUALIZACIÓN plot(0:100 , p\_heads(0:100, 100, 0.5), type =
``h'') \# f(x) de probabilidad \#pintar desviación típica: abline(v = 50
, col = 2, lwd = 3) \# dibuja linea vertical en el 50 (lwd = line width)
abline( v = 50 - sd\_heads , col = 2, lwd = 3) \#me muevo izq tantas
veces como me diga la desviación abline( v = 50 + sd\_heads , col = 2,
lwd = 3) \# me muevo a la dcha \\
\#(3 a la izqœ y 3 a la dcha cubre el 99\% de los casos) abline( v = 50
- 3 \emph{sd\_heads , col = 3, lwd = 3) abline( v = 50 + 3 }sd\_heads ,
col = 3, lwd = 3) ``` \\
\includegraphics{3-random_vars_p1_statements_files/figure-latex/unnamed-chunk-6-1.pdf} \\
\texttt{r\ \#\ 2)\ SIMULACIONES\ N\ =\ 5000\ xs\ =\ replicate\ (\ N\ ,\ \{\ sum(sample(0:1,\ replace\ =\ "TRUE",\ 100))\ \#sumila\ resultado\ aleatorio,\ lanzar\ 100\ veces\ moneda\ \})\ sum(xs)/N\ \ \#E{[}X{]}===\textgreater{}\ generar\ mazo\ xs,\ mean(xs)} \\
\texttt{\#\#\ {[}1{]}\ 49.872} \\
\texttt{r\ mean((xs\ -50)\^{}2)\ \#Var{[}x{]}\ ===\textgreater{}\ generar\ xs,\ generar\ la\ cantidad\ q\ m\ interesa((xs\ -\ mu)\^{}2),\ hacer\ mean\ de\ eso} \\
\texttt{\#\#\ {[}1{]}\ 26.4728} \\
Un jugador gana 1 euro si al tirar un dado obtiene un 1 o un 3; pierde 2
euros si sale un 2, 4, 6; y gana 4 euros si sale un 5. ¿Cuál es la
ganancia esperada? ¿Jugarías a este juego? \\
\texttt{r\ \#\ ???} \\
\# Distribuciones conjuntas de variables discretas \#\# Distribuciones
conjuntas de variables discretas \\
\bottomrule()
\end{longtable}

Se lanza una moneda \(n\) veces (prob. de cara es \(p\)). Considera las
VAs X: `'nº de caras'' e Y: `'nº de caras iniciales (antes de la primera
cruz o del fin del experimento)''. Halla la distribución conjunta para
cualquier \(n\) y \(p\) y luego particulariza para \(n=4\),\(p=0.5\).

\begin{Shaded}
\begin{Highlighting}[]
\CommentTok{\# ???}
\end{Highlighting}
\end{Shaded}

\begin{center}\rule{0.5\linewidth}{0.5pt}\end{center}

Partiendo de la función de probabilidad conjunta del ejemplo anterior,
(\(n=4\) \(p=0.5\)), calcula las funciones de probabilidad marginales
para X (nº de caras) e Y (nº de caras iniciales).

\begin{Shaded}
\begin{Highlighting}[]
\CommentTok{\# ???}
\end{Highlighting}
\end{Shaded}

\hypertarget{distribuciones-condicionales-de-variables-discretas}{%
\subsection{Distribuciones condicionales de variables
discretas}\label{distribuciones-condicionales-de-variables-discretas}}

\begin{center}\rule{0.5\linewidth}{0.5pt}\end{center}

En una urna hay dos monedas trucadas con probabilidad de cara
\(p_0=0.4\) y \(p_1=0.6\). Se elige una al azar y se tira 100 veces. Sea
X:`'nº de caras obtenidas'' e Y: `'moneda elegida''. Obtener la función
de probabilidad de X.

\begin{Shaded}
\begin{Highlighting}[]
\CommentTok{\# ???}
\end{Highlighting}
\end{Shaded}

\begin{center}\rule{0.5\linewidth}{0.5pt}\end{center}

Si se han obtenido X=48 caras, ¿cuál es la probabilidad de que la moneda
usada sea la correspondiente a \(p_0\)?

\begin{Shaded}
\begin{Highlighting}[]
\CommentTok{\# ???}
\end{Highlighting}
\end{Shaded}

\hypertarget{estaduxedsticos-de-variables-aleatorias-conjuntas}{%
\subsection{Estadísticos de variables aleatorias
conjuntas}\label{estaduxedsticos-de-variables-aleatorias-conjuntas}}

Sea X:`'Cantidad mensual de lotes comprados por una empresa a su
proveedor'' e Y: `'Precio por lote ofertado por el proveedor (en miles
de euros)''. La distribución conjunta de ambas variables se recoge en la
siguiente tabla:

\begin{Shaded}
\begin{Highlighting}[]
\NormalTok{probs }\OtherTok{=} \FunctionTok{matrix}\NormalTok{(}
  \FunctionTok{c}\NormalTok{(}\FloatTok{0.00}\NormalTok{, }\FloatTok{0.00}\NormalTok{, }\FloatTok{0.03}\NormalTok{, }\FloatTok{0.18}\NormalTok{,}
    \FloatTok{0.00}\NormalTok{, }\FloatTok{0.04}\NormalTok{, }\FloatTok{0.24}\NormalTok{, }\FloatTok{0.02}\NormalTok{,}
    \FloatTok{0.02}\NormalTok{, }\FloatTok{0.23}\NormalTok{, }\FloatTok{0.04}\NormalTok{, }\FloatTok{0.00}\NormalTok{,}
    \FloatTok{0.16}\NormalTok{, }\FloatTok{0.04}\NormalTok{, }\FloatTok{0.00}\NormalTok{, }\FloatTok{0.00}\NormalTok{),}
  \AttributeTok{byrow =} \ConstantTok{TRUE}\NormalTok{,}
  \AttributeTok{ncol =} \DecValTok{4}
\NormalTok{)}
\FunctionTok{rownames}\NormalTok{(probs) }\OtherTok{\textless{}{-}} \FunctionTok{paste0}\NormalTok{(}\StringTok{"y="}\NormalTok{, }\FunctionTok{seq}\NormalTok{(}\DecValTok{1}\NormalTok{, }\FloatTok{2.5}\NormalTok{, }\AttributeTok{by=}\FloatTok{0.5}\NormalTok{))}
\FunctionTok{colnames}\NormalTok{(probs) }\OtherTok{\textless{}{-}} \FunctionTok{paste0}\NormalTok{(}\StringTok{"x="}\NormalTok{, }\DecValTok{1}\SpecialCharTok{:}\DecValTok{4}\NormalTok{)}
\end{Highlighting}
\end{Shaded}

¿Cuál es el coste esperado para la empresa en el siguiente mes de
actividad?

\begin{Shaded}
\begin{Highlighting}[]
\CommentTok{\# ???}
\end{Highlighting}
\end{Shaded}

\begin{center}\rule{0.5\linewidth}{0.5pt}\end{center}

Calcula la correlación entre X e Y en el problema de la empresa y el
proveedor.

\begin{Shaded}
\begin{Highlighting}[]
\CommentTok{\# ???}
\end{Highlighting}
\end{Shaded}


\end{document}
